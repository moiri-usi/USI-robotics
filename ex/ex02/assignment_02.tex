%%%%%%%%%%%%%%%%%%%%%%%%%%%%%%%%%%%%%%%%%
% Short Sectioned Assignment
% LaTeX Template
% Version 1.0 (5/5/12)
%
% This template has been downloaded from:
% http://www.LaTeXTemplates.com
%
% Original author:
% Frits Wenneker (http://www.howtotex.com)
%
% License:
% CC BY-NC-SA 3.0 (http://creativecommons.org/licenses/by-nc-sa/3.0/)
%
%%%%%%%%%%%%%%%%%%%%%%%%%%%%%%%%%%%%%%%%%

%----------------------------------------------------------------------------------------
%	PACKAGES AND OTHER DOCUMENT CONFIGURATIONS
%----------------------------------------------------------------------------------------

\documentclass[paper=a4, fontsize=11pt]{scrartcl} % A4 paper and 11pt font size

\usepackage[utf8]{inputenc}
%\usepackage[T1]{fontenc} % Use 8-bit encoding that has 256 glyphs
\usepackage{fourier} % Use the Adobe Utopia font for the document - comment this line to return to the LaTeX default
\usepackage[english]{babel} % English language/hyphenation
\usepackage{amsmath,amsfonts,amsthm} % Math packages

\usepackage{nameref}
\usepackage{graphicx} % Required to insert images
%\usepackage{lipsum} % Used for inserting dummy 'Lorem ipsum' text into the template

\usepackage{sectsty} % Allows customizing section commands
\allsectionsfont{\centering \normalfont\scshape} % Make all sections centered, the default font and small caps

\usepackage{fancyhdr} % Custom headers and footers
\pagestyle{fancyplain} % Makes all pages in the document conform to the custom headers and footers
\fancyhead{} % No page header - if you want one, create it in the same way as the footers below
\fancyfoot[L]{} % Empty left footer
\fancyfoot[C]{} % Empty center footer
\fancyfoot[R]{\thepage} % Page numbering for right footer
\renewcommand{\headrulewidth}{0pt} % Remove header underlines
\renewcommand{\footrulewidth}{0pt} % Remove footer underlines
\setlength{\headheight}{13.6pt} % Customize the height of the header

%\numberwithin{equation}{section} % Number equations within sections (i.e. 1.1, 1.2, 2.1, 2.2 instead of 1, 2, 3, 4)
%\numberwithin{figure}{section} % Number figures within sections (i.e. 1.1, 1.2, 2.1, 2.2 instead of 1, 2, 3, 4)
%\numberwithin{table}{section} % Number tables within sections (i.e. 1.1, 1.2, 2.1, 2.2 instead of 1, 2, 3, 4)

\setlength\parindent{0pt} % Removes all indentation from paragraphs - comment this line for an assignment with lots of text

%----------------------------------------------------------------------------------------
%	TITLE SECTION
%----------------------------------------------------------------------------------------

\newcommand{\horrule}[1]{\rule{\linewidth}{#1}} % Create horizontal rule command with 1 argument of height

\title{	
\normalfont \normalsize 
\textsc{Università della Svizzera Italiana, Faculty of Informatics} \\ [25pt] % Your university, school and/or department name(s)
\horrule{0.5pt} \\[0.4cm] % Thin top horizontal rule
\huge Mobile Robot Kinematics \\ % The assignment title
\horrule{2pt} \\[0.5cm] % Thick bottom horizontal rule
}

\author{Simon Maurer} % Your name

\date{\normalsize\today} % Today's date or a custom date

\begin{document}

\maketitle % Print the title

%----------------------------------------------------------------------------------------
%	PROBLEM 1
%----------------------------------------------------------------------------------------

\section{Assignment 2.1}

Differential drive robot:
\begin{itemize}
    \item Distance between wheels $ d_r = 2r = 60 mm $
    \item Wheel diameter $ d_w = 40 mm $
    \item 1000 steps per revolution of the wheel
\end{itemize}

%------------------------------------------------

\subsection{First 10 seconds}
\begin{itemize}
    \item left wheel is rotating with 375 steps per second
    \item right wheel is rotating with 150 steps per second
    \item starting at the pose $ (x, y, \theta) = (0, 0, 0) $
\end{itemize}

Calculation of the wheel speeds:
\begin{equation}
    v_l = 2\pi * \frac{375}{375} * \frac{d_w}{2} = 40\pi \frac{mm}{s}
\end{equation}
\begin{equation}
    v_r = 2\pi * \frac{150}{375} * \frac{d_w}{2} = 16\pi \frac{mm}{s}
\end{equation}
Calculation of $ \omega $, $ R $ and $ ICC $:
\begin{equation}
    \omega = \frac{v_r - v_l}{2r} = -0.4\pi \frac{rad}{s}
\end{equation}
\begin{equation}
    R = \frac{r(v_r + v_l)}{v_r - v_l} = -70 mm
\end{equation}
\begin{equation}
    ICC = \begin{bmatrix}
        x - R\sin(\theta)\\
        y + R\cos(\theta)
    \end{bmatrix} = \begin{bmatrix}
        0\\
        R
    \end{bmatrix} = \begin{bmatrix}
        0\\
        -70 mm
    \end{bmatrix}
\end{equation}
Calculation of the new pose (with $ \delta t = 10 s) $:
\begin{equation}
    \begin{bmatrix}
        x'\\
        y'\\
        \theta'
    \end{bmatrix}
    = \begin{bmatrix}
        \cos(\omega \delta t) & -\sin(\omega \delta t) & 0\\
        \sin(\omega \delta t) & \cos(\omega \delta t) & 0\\
        0 & 0 & 1
    \end{bmatrix}
    \begin{bmatrix}
        x - ICC_x\\
        y - ICC_y\\
        \theta
    \end{bmatrix}
    + \begin{bmatrix}
        ICC_x\\
        ICC_y\\
        \omega \delta t
    \end{bmatrix}
    = \begin{bmatrix}
        0\\
        0\\
        0
    \end{bmatrix}
\end{equation}
The robot turns two times clockwise around the point $ (0, -70 mm) $ and ends
after 10 seconds at the origin.

%------------------------------------------------

\subsection{Following 20 seconds}

\begin{itemize}
    \item left wheel is rotating with 600 steps per second
    \item right wheel is rotating with 300 steps per second
    \item starting at the pose $ (x, y, \theta) = (0, 0, 0) $ (as calcualated
        before)
\end{itemize}

Calculation of the wheel speeds:
\begin{equation}
    v_l = 2\pi * \frac{600}{375} * \frac{d_w}{2} = 64\pi \frac{mm}{s}
\end{equation}
\begin{equation}
    v_r = 2\pi * \frac{300}{375} * \frac{d_w}{2} = 32\pi \frac{mm}{s}
\end{equation}
Calculation of $ \omega $, $ R $ and $ ICC $:
\begin{equation}
    \omega = \frac{v_r - v_l}{2r} = -\frac{8}{15}\pi \frac{rad}{s}
\end{equation}
\begin{equation}
    R = \frac{r(v_r + v_l)}{v_r - v_l} = -60 mm
\end{equation}
\begin{equation}
    ICC = \begin{bmatrix}
        x - R\sin(\theta)\\
        y + R\cos(\theta)
    \end{bmatrix} = \begin{bmatrix}
        0\\
        R
    \end{bmatrix} = \begin{bmatrix}
        0\\
        -60 mm
    \end{bmatrix}
\end{equation}
Calculation of the new pose (with $ \delta t = 20 s) $:
\begin{equation}
    \begin{bmatrix}
        x'\\
        y'\\
        \theta'
    \end{bmatrix}
    = \begin{bmatrix}
        \cos(\omega \delta t) & -\sin(\omega \delta t) & 0\\
        \sin(\omega \delta t) & \cos(\omega \delta t) & 0\\
        0 & 0 & 1
    \end{bmatrix}
    \begin{bmatrix}
        x - ICC_x\\
        y - ICC_y\\
        \theta
    \end{bmatrix}
    + \begin{bmatrix}
        ICC_x\\
        ICC_y\\
        \omega \delta t
    \end{bmatrix}
    = \begin{bmatrix}
        30\sqrt 3 mm\\
        -90 mm\\
        -\frac{2}{3} \pi
    \end{bmatrix}
\end{equation}
The robot passes fife times the origin while turning clockwise around the point
$ (0, -60 mm) $ and then stops at the pose $ (52 mm, -90 mm, -\frac{2}{3} \pi)
$.
%------------------------------------------------

%----------------------------------------------------------------------------------------
%	PROBLEM 2
%----------------------------------------------------------------------------------------

\section{Assignment 2.2}

Holonomic drive robot
\begin{itemize}
\item Infinite torque
\item Maximum speed of $ \pm 1 \frac{m}{s} $
\end{itemize}

When driving forward with maximal speed ($ v_x = 0, v_y = v_{max}, \omega
= 0 $), the speed of the rear wheel is zero and the speed of the two front
wheels (arranged in an angle of 120 degrees) is maximal.
\begin{equation}
    \label{inv_kin}
    \begin{bmatrix}
        v_x\\
        v_y\\
        r \omega
    \end{bmatrix}
    = \begin{bmatrix}
        -\sin(\varphi_1) & \cos(\varphi_1) & 1\\
        -\sin(\varphi_2) & \cos(\varphi_2) & 1\\
        -\sin(\varphi_3) & \cos(\varphi_3) & 1\\
    \end{bmatrix}^+
    \begin{bmatrix}
        v_1\\
        v_2\\
        v_3
    \end{bmatrix}
\end{equation}
where matrix $ A^+ $ is the pseudoinverse of matrix $ A $ and
\begin{equation}
    \label{v}
    \begin{bmatrix}
        v_1\\
        v_2\\
        v_3
    \end{bmatrix}
    = \begin{bmatrix}
        1\\
        -1\\
        0
    \end{bmatrix} \frac{m}{s}
\end{equation}
\begin{equation}
    \label{angles_1}
    \begin{bmatrix}
        \varphi_1\\
        \varphi_2\\
        \varphi_3
    \end{bmatrix}
    = \begin{bmatrix}
        \frac{1}{6}\pi\\
        \frac{5}{6}\pi\\
        \frac{3}{2}\pi
    \end{bmatrix}
\end{equation}
With equation \ref{inv_kin} and the values \ref{v} and \ref{angles_1} we can now calculate
\begin{equation}
    \begin{bmatrix}
        v_x\\
        v_y\\
        r \omega
    \end{bmatrix}
    = \begin{bmatrix}
        0\\
        1.15\\
        0
    \end{bmatrix}
\end{equation}
hence $ v_y = v_{max} = 1.15 \frac{m}{s}$.

By arranging the front wheels in an angle of 90 degrees we use
\begin{equation}
    \label{angles_2}
    \begin{bmatrix}
        \varphi_1\\
        \varphi_2\\
        \varphi_3
    \end{bmatrix}
    = \begin{bmatrix}
        \frac{1}{4}\pi\\
        \frac{3}{4}\pi\\
        \frac{3}{2}\pi
    \end{bmatrix}
\end{equation}
and again with equation \ref{inv_kin} and the values \ref{v} and \ref{angles_2} we can calculate
\begin{equation}
    \begin{bmatrix}
        v_x\\
        v_y\\
        r \omega
    \end{bmatrix}
    = \begin{bmatrix}
        0\\
        1.41\\
        0
    \end{bmatrix}
\end{equation}
hence $ v_y = v_{max} = 1.41 \frac{m}{s}$.

We can see, that with an angle of 90 degrees, the robot is moving faster in
direction y. By arranging the front wheels even closer we can in theory
increase the speed of the robot to infinity. This is represented on figure
\ref{fig:speed}. In reality, this of course is not possible because infinite
torque is pretty hard to come by.
\begin{figure}[h]
    \includegraphics[width=1\columnwidth]{graph_02}
    \caption{Forward speed of the robot in function of the angle between the
        front wheels}\label{fig:speed}
\end{figure}
\end{document}
