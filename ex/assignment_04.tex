%%%%%%%%%%%%%%%%%%%%%%%%%%%%%%%%%%%%%%%%%
% Short Sectioned Assignment
% LaTeX Template
% Version 1.0 (5/5/12)
%
% This template has been downloaded from:
% http://www.LaTeXTemplates.com
%
% Original author:
% Frits Wenneker (http://www.howtotex.com)
%
% License:
% CC BY-NC-SA 3.0 (http://creativecommons.org/licenses/by-nc-sa/3.0/)
%
%%%%%%%%%%%%%%%%%%%%%%%%%%%%%%%%%%%%%%%%%

%----------------------------------------------------------------------------------------
%	PACKAGES AND OTHER DOCUMENT CONFIGURATIONS
%----------------------------------------------------------------------------------------

\documentclass[paper=a4, fontsize=11pt]{scrartcl} % A4 paper and 11pt font size

\usepackage[utf8]{inputenc}
%\usepackage[T1]{fontenc} % Use 8-bit encoding that has 256 glyphs
\usepackage{fourier} % Use the Adobe Utopia font for the document - comment this line to return to the LaTeX default
\usepackage[english]{babel} % English language/hyphenation
\usepackage{amsmath,amsfonts,amsthm} % Math packages

\usepackage{nameref}
\usepackage{graphicx} % Required to insert images
%\usepackage{lipsum} % Used for inserting dummy 'Lorem ipsum' text into the template

\usepackage{sectsty} % Allows customizing section commands
\allsectionsfont{\centering \normalfont\scshape} % Make all sections centered, the default font and small caps

\usepackage{fancyhdr} % Custom headers and footers
\pagestyle{fancyplain} % Makes all pages in the document conform to the custom headers and footers
\fancyhead{} % No page header - if you want one, create it in the same way as the footers below
\fancyfoot[L]{} % Empty left footer
\fancyfoot[C]{} % Empty center footer
\fancyfoot[R]{\thepage} % Page numbering for right footer
\renewcommand{\headrulewidth}{0pt} % Remove header underlines
\renewcommand{\footrulewidth}{0pt} % Remove footer underlines
\setlength{\headheight}{13.6pt} % Customize the height of the header

%\numberwithin{equation}{section} % Number equations within sections (i.e. 1.1, 1.2, 2.1, 2.2 instead of 1, 2, 3, 4)
%\numberwithin{figure}{section} % Number figures within sections (i.e. 1.1, 1.2, 2.1, 2.2 instead of 1, 2, 3, 4)
%\numberwithin{table}{section} % Number tables within sections (i.e. 1.1, 1.2, 2.1, 2.2 instead of 1, 2, 3, 4)

\setlength\parindent{0pt} % Removes all indentation from paragraphs - comment this line for an assignment with lots of text

%----------------------------------------------------------------------------------------
%	TITLE SECTION
%----------------------------------------------------------------------------------------

\newcommand{\horrule}[1]{\rule{\linewidth}{#1}} % Create horizontal rule command with 1 argument of height

\title{	
\normalfont \normalsize 
\textsc{Università della Svizzera Italiana, Faculty of Informatics} \\ [25pt] % Your university, school and/or department name(s)
\horrule{0.5pt} \\[0.4cm] % Thin top horizontal rule
\huge Kalman Filter \\ % The assignment title
\horrule{2pt} \\[0.5cm] % Thick bottom horizontal rule
}

\author{Simon Maurer} % Your name

\date{\normalsize\today} % Today's date or a custom date

\begin{document}

\maketitle % Print the title

%----------------------------------------------------------------------------------------
%	PROBLEM 1
%----------------------------------------------------------------------------------------

\section{Assignment 4.1}

Figure \ref{fig:filter} shows in the first row the estimated model states, using
the kalman filter. In the second and the third row the estimated model states
are represented, using the moving average filter. Four different window sizes
have been used (1, 2, 3, 4). Note: A window size of 1 corresponds to the
original sensor data.

By doing a visual analysis, one would claim, that the kalman filter gives the
best result, followed be the moving average filter. The bigger the window size
of the moving average filter, the better seems the result because the graph is
smoother.

Calculating the mean square error gives additional information about the
quality. While the kalman filter gives still the best result, the moving
average with a window size of 2 is now better than the one with a window size
of 4 (as claimed above). An increasing window size has indeed a smoothing
effect on the graph and can improve the precision of the estimated model states
but by increasing the size to much, the information loss becomes more important
than the smoothing and hence the error increases.

Following the calculated mean square errors:
\begin{itemize}
    \item $ e_{kal} = 0.00265 $
    \item $ e_{av1} = 0.00875 $
    \item $ e_{av2} = 0.00567 $
    \item $ e_{av3} = 0.00742 $
    \item $ e_{av4} = 0.01146 $
\end{itemize}

\begin{figure}[h]
    \includegraphics[width=1\columnwidth]{ex04_graph}
    \caption{Kalman Filter vs Moving Average with different window sizes}
    \label{fig:filter}
\end{figure}
\end{document}
