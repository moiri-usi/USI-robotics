%%%%%%%%%%%%%%%%%%%%%%%%%%%%%%%%%%%%%%%%%
% Short Sectioned Assignment
% LaTeX Template
% Version 1.0 (5/5/12)
%
% This template has been downloaded from:
% http://www.LaTeXTemplates.com
%
% Original author:
% Frits Wenneker (http://www.howtotex.com)
%
% License:
% CC BY-NC-SA 3.0 (http://creativecommons.org/licenses/by-nc-sa/3.0/)
%
%%%%%%%%%%%%%%%%%%%%%%%%%%%%%%%%%%%%%%%%%

%----------------------------------------------------------------------------------------
%	PACKAGES AND OTHER DOCUMENT CONFIGURATIONS
%----------------------------------------------------------------------------------------

\documentclass[paper=a4, fontsize=11pt]{scrartcl} % A4 paper and 11pt font size

\usepackage[utf8]{inputenc}
%\usepackage[T1]{fontenc} % Use 8-bit encoding that has 256 glyphs
\usepackage{fourier} % Use the Adobe Utopia font for the document - comment this line to return to the LaTeX default
\usepackage[english]{babel} % English language/hyphenation
\usepackage{amsmath,amsfonts,amsthm} % Math packages

\usepackage{nameref}
\usepackage{graphicx} % Required to insert images
%\usepackage{lipsum} % Used for inserting dummy 'Lorem ipsum' text into the template

\usepackage{sectsty} % Allows customizing section commands
\allsectionsfont{\centering \normalfont\scshape} % Make all sections centered, the default font and small caps

\usepackage{fancyhdr} % Custom headers and footers
\pagestyle{fancyplain} % Makes all pages in the document conform to the custom headers and footers
\fancyhead{} % No page header - if you want one, create it in the same way as the footers below
\fancyfoot[L]{} % Empty left footer
\fancyfoot[C]{} % Empty center footer
\fancyfoot[R]{\thepage} % Page numbering for right footer
\renewcommand{\headrulewidth}{0pt} % Remove header underlines
\renewcommand{\footrulewidth}{0pt} % Remove footer underlines
\setlength{\headheight}{13.6pt} % Customize the height of the header

%\numberwithin{equation}{section} % Number equations within sections (i.e. 1.1, 1.2, 2.1, 2.2 instead of 1, 2, 3, 4)
%\numberwithin{figure}{section} % Number figures within sections (i.e. 1.1, 1.2, 2.1, 2.2 instead of 1, 2, 3, 4)
%\numberwithin{table}{section} % Number tables within sections (i.e. 1.1, 1.2, 2.1, 2.2 instead of 1, 2, 3, 4)

\setlength\parindent{0pt} % Removes all indentation from paragraphs - comment this line for an assignment with lots of text

%----------------------------------------------------------------------------------------
%	TITLE SECTION
%----------------------------------------------------------------------------------------

\newcommand{\horrule}[1]{\rule{\linewidth}{#1}} % Create horizontal rule command with 1 argument of height

\title{	
\normalfont \normalsize 
\textsc{Università della Svizzera Italiana, Faculty of Informatics} \\ [25pt] % Your university, school and/or department name(s)
\horrule{0.5pt} \\[0.4cm] % Thin top horizontal rule
\huge Motors and Locomotion \\ % The assignment title
\horrule{2pt} \\[0.5cm] % Thick bottom horizontal rule
}

\author{Simon Maurer} % Your name

\date{\normalsize\today} % Today's date or a custom date

\begin{document}

\maketitle % Print the title

%----------------------------------------------------------------------------------------
%	PROBLEM 1
%----------------------------------------------------------------------------------------

\section{Assignment I}

E-Puck robot (with stepper motors)
\begin{itemize}
\item Distance between wheels 60 mm
\item Wheel diameter 40 mm
\item 20 ticks per revolution (motor shaft)
\item 50:1 gearbox reduction ratio (50 rotations of the motor results in 1 rotation of the gear shaft)
\end{itemize}

The wheel circumference is $ u_w = \pi * 40 mm $, the number of ticks necessary
to complete one wheel revolution is $ t_w = 20 * 50 = 1000 $. The circumference
of the robot turning on the spot is $ u_r = \pi * 60 mm $.
%------------------------------------------------

\subsection{Turn on Place}
The distance a wheel must travel in order to turn the robot by 90° is
$ d = \frac{u_r}{4} mm $. To travel this distance, $ n = \frac{d}{u_w} $ wheel
turns are necessary. Finally, the number of ticks necessary to turn the wheel
$ n $ times are $ n * t_w = \frac{u_r}{4 * u_w} * t_w = 375 $ ticks.


%------------------------------------------------

\subsection{Drive Forward}
Now the distance a wheel needs to travel is $ d = 100 mm $. The number of ticks
necessary to travel $ d $ is $ \frac{d}{u_w} * t_w = 796 $ ticks.

%------------------------------------------------

%----------------------------------------------------------------------------------------
%	PROBLEM 2
%----------------------------------------------------------------------------------------

\section{Assignment II}

Thymio II robot (with DC motors)
\begin{itemize}
\item Distance between wheels 94 mm
\item Wheel diameter 44 mm
\end{itemize}

The wheel circumference is $ u_w = \pi * 44 mm $.
%------------------------------------------------

\subsection{Revolution of the Wheel}
\label{rev}
To be able to associate the integer number of the motor speed to a speed of the
robot, measurements have been performed. Each blue point of Figure
\ref{fig:dist} shows the distance the robot has traveled during 5 seconds at
a corresponding motor speed. At a motor speed lower than 400, the traveled
distance increases linearly. At a motor speed of 500 however, the traveled
distance is lower than expected. One possible explanation is slippage of the
robot at a higher speed. Another explanation is related to the red graph on
Figure \ref{fig:dist}: The red points represent the deviation from the intended
straight traveling line of the robot, i.e. indicates a motor speed difference
of the two wheels. Such a difference and the resulting turning of the robot may
have a large impact on the traveling distance of the robot (probably due to
increased friction).
\begin{figure}[h]
    \includegraphics[width=1\columnwidth]{graph} % Example image
    \caption{Traveling distance in function of the motor speed}\label{fig:dist}
\end{figure}

%------------------------------------------------

\subsection{Draw a Circle}
\paragraph{Theoretical Approach}
To draw a circle of $ R = 10 cm $ with the robot, the inner wheel will describe
a radius $ r_i = R - r = 100 mm - \frac{94}{2} mm = 53 mm $ and the outer wheel
a radius $ r_o = R + r = 100 mm + \frac{94}{2} mm = 147 mm $. The relation
between the target speed of the two wheels is then $ \frac{v_o}{v_i}
= \frac{r_o}{r_i} = \frac{R + r}{R - r} = 2.77 $.
\paragraph{Experiment}
According to the measurements of the problem \nameref{rev}, a speed of 300
seems reasonable because the wheels turn with almost the same speed. For the
experiment the speed of the inner wheel is set to $ v_i = 200 $ and the speed
for the outer wheel is set to $ v_o = 560 $. With these parameters, the robot
describes a circle with radius $ R' = 105 \pm 2 mm $. To correct this, in
practice the relation of the speed of the wheels should be $ \frac{v_o}{v_i}
= \frac{R' + r}{R' - r} = 2.62 $
\end{document}
